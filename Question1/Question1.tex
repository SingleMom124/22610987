\documentclass[11pt,preprint, authoryear]{elsarticle}

\usepackage{lmodern}
%%%% My spacing
\usepackage{setspace}
\setstretch{1.2}
\DeclareMathSizes{12}{14}{10}{10}

% Wrap around which gives all figures included the [H] command, or places it "here". This can be tedious to code in Rmarkdown.
\usepackage{float}
\let\origfigure\figure
\let\endorigfigure\endfigure
\renewenvironment{figure}[1][2] {
    \expandafter\origfigure\expandafter[H]
} {
    \endorigfigure
}

\let\origtable\table
\let\endorigtable\endtable
\renewenvironment{table}[1][2] {
    \expandafter\origtable\expandafter[H]
} {
    \endorigtable
}


\usepackage{ifxetex,ifluatex}
\usepackage{fixltx2e} % provides \textsubscript
\ifnum 0\ifxetex 1\fi\ifluatex 1\fi=0 % if pdftex
  \usepackage[T1]{fontenc}
  \usepackage[utf8]{inputenc}
\else % if luatex or xelatex
  \ifxetex
    \usepackage{mathspec}
    \usepackage{xltxtra,xunicode}
  \else
    \usepackage{fontspec}
  \fi
  \defaultfontfeatures{Mapping=tex-text,Scale=MatchLowercase}
  \newcommand{\euro}{€}
\fi

\usepackage{amssymb, amsmath, amsthm, amsfonts}

\def\bibsection{\section*{References}} %%% Make "References" appear before bibliography


\usepackage[round]{natbib}

\usepackage{longtable}
\usepackage[margin=2.3cm,bottom=2cm,top=2.5cm, includefoot]{geometry}
\usepackage{fancyhdr}
\usepackage[bottom, hang, flushmargin]{footmisc}
\usepackage{graphicx}
\numberwithin{equation}{section}
\numberwithin{figure}{section}
\numberwithin{table}{section}
\setlength{\parindent}{0cm}
\setlength{\parskip}{1.3ex plus 0.5ex minus 0.3ex}
\usepackage{textcomp}
\renewcommand{\headrulewidth}{0.2pt}
\renewcommand{\footrulewidth}{0.3pt}

\usepackage{array}
\newcolumntype{x}[1]{>{\centering\arraybackslash\hspace{0pt}}p{#1}}

%%%%  Remove the "preprint submitted to" part. Don't worry about this either, it just looks better without it:
\makeatletter
\def\ps@pprintTitle{%
  \let\@oddhead\@empty
  \let\@evenhead\@empty
  \let\@oddfoot\@empty
  \let\@evenfoot\@oddfoot
}
\makeatother

 \def\tightlist{} % This allows for subbullets!

\usepackage{hyperref}
\hypersetup{breaklinks=true,
            bookmarks=true,
            colorlinks=true,
            citecolor=blue,
            urlcolor=blue,
            linkcolor=blue,
            pdfborder={0 0 0}}


% The following packages allow huxtable to work:
\usepackage{siunitx}
\usepackage{multirow}
\usepackage{hhline}
\usepackage{calc}
\usepackage{tabularx}
\usepackage{booktabs}
\usepackage{caption}


\newenvironment{columns}[1][]{}{}

\newenvironment{column}[1]{\begin{minipage}{#1}\ignorespaces}{%
\end{minipage}
\ifhmode\unskip\fi
\aftergroup\useignorespacesandallpars}

\def\useignorespacesandallpars#1\ignorespaces\fi{%
#1\fi\ignorespacesandallpars}

\makeatletter
\def\ignorespacesandallpars{%
  \@ifnextchar\par
    {\expandafter\ignorespacesandallpars\@gobble}%
    {}%
}
\makeatother

\newenvironment{CSLReferences}[2]{%
}

\urlstyle{same}  % don't use monospace font for urls
\setlength{\parindent}{0pt}
\setlength{\parskip}{6pt plus 2pt minus 1pt}
\setlength{\emergencystretch}{3em}  % prevent overfull lines
\setcounter{secnumdepth}{5}

%%% Use protect on footnotes to avoid problems with footnotes in titles
\let\rmarkdownfootnote\footnote%
\def\footnote{\protect\rmarkdownfootnote}
\IfFileExists{upquote.sty}{\usepackage{upquote}}{}

%%% Include extra packages specified by user
\usepackage{array}
\usepackage{caption}
\usepackage{graphicx}
\usepackage{siunitx}
\usepackage[normalem]{ulem}
\usepackage{colortbl}
\usepackage{multirow}
\usepackage{hhline}
\usepackage{calc}
\usepackage{tabularx}
\usepackage{threeparttable}
\usepackage{wrapfig}
\usepackage{adjustbox}
\usepackage{hyperref}

%%% Hard setting column skips for reports - this ensures greater consistency and control over the length settings in the document.
%% page layout
%% paragraphs
\setlength{\baselineskip}{12pt plus 0pt minus 0pt}
\setlength{\parskip}{12pt plus 0pt minus 0pt}
\setlength{\parindent}{0pt plus 0pt minus 0pt}
%% floats
\setlength{\floatsep}{12pt plus 0 pt minus 0pt}
\setlength{\textfloatsep}{20pt plus 0pt minus 0pt}
\setlength{\intextsep}{14pt plus 0pt minus 0pt}
\setlength{\dbltextfloatsep}{20pt plus 0pt minus 0pt}
\setlength{\dblfloatsep}{14pt plus 0pt minus 0pt}
%% maths
\setlength{\abovedisplayskip}{12pt plus 0pt minus 0pt}
\setlength{\belowdisplayskip}{12pt plus 0pt minus 0pt}
%% lists
\setlength{\topsep}{10pt plus 0pt minus 0pt}
\setlength{\partopsep}{3pt plus 0pt minus 0pt}
\setlength{\itemsep}{5pt plus 0pt minus 0pt}
\setlength{\labelsep}{8mm plus 0mm minus 0mm}
\setlength{\parsep}{\the\parskip}
\setlength{\listparindent}{\the\parindent}
%% verbatim
\setlength{\fboxsep}{5pt plus 0pt minus 0pt}



\begin{document}



\begin{frontmatter}  %

\title{Cultural Dynamics and Persistence in Baby Naming Trends: An
Analysis of Influences and Longevity in the United States}

% Set to FALSE if wanting to remove title (for submission)




\author[Add1]{Joshua-Connor Knapp\footnote{\textbf{Contributions:}
  \newline \emph{So, how about that airline food?.}}}
\ead{joshuaconnok@gmail.com}





\address[Add1]{Flex Offenders, Local Garage Gym, South Africa}

\cortext[cor]{Corresponding author: Joshua-Connor Knapp\footnote{\textbf{Contributions:}
  \newline \emph{So, how about that airline food?.}}}

\begin{abstract}
\small{
This report explores the evolving landscape of baby naming trends in the
United States, highlighting the influence of cultural factors such as
popular media and prominent public figures. By analyzing data spanning
from 1910 to 2014, we examine the persistence of popular names and
identify gender-specific patterns. Our findings reveal a significant
decline in the persistence of boys' names since the 1990s, contrasted
with the resilience of girls' names. Additionally, we investigate the
impact of actor names on naming conventions using regression models,
uncovering substantial correlations between popular actor names and
increased name counts in movie titles. These insights underscore the
interplay between cultural dynamics and societal preferences in shaping
naming trends.
}
\end{abstract}

\vspace{1cm}





\vspace{0.5cm}

\end{frontmatter}

\setcounter{footnote}{0}



%________________________
% Header and Footers
%%%%%%%%%%%%%%%%%%%%%%%%%%%%%%%%%
\pagestyle{fancy}
\chead{}
\rhead{}
\lfoot{}
\rfoot{\footnotesize Page \thepage}
\lhead{}
%\rfoot{\footnotesize Page \thepage } % "e.g. Page 2"
\cfoot{}

%\setlength\headheight{30pt}
%%%%%%%%%%%%%%%%%%%%%%%%%%%%%%%%%
%________________________

\headsep 35pt % So that header does not go over title




\section{Introduction}\label{introduction}

The realm of baby naming trends in the United States has evolved
significantly over the years, influenced by a myriad of factors ranging
from popular culture to historical events. Understanding these
influences not only provides insights into societal trends but also
reflects broader cultural shifts. As we delve into the data, we observe
intriguing patterns such as the impact of popular movie characters,
prominent political figures, celebrities, and even musical sensations on
naming practices. Moreover, analyzing the longevity of these trends
reveals whether certain names endure over generations or whether others
quickly fade into obscurity. Interestingly, recent findings suggest that
while overall naming trends may be stabilizing, gender-specific analyses
unveil distinct patterns influenced by factors like actor names. This
intersection of data analytics and cultural influence offers a
fascinating lens through which to explore the dynamic landscape of baby
naming in the US.

\section{Data}\label{data}

This report primarily considers, three data sets. All take the form of
panel data. The first details baby naming conventions, providing the the
count of a particular name for years ranging from 1910 to 2014. The
other two data sets are comprised of actor credits and titles released
by HBO. Given that both of these data sets shared an ID column referring
to a specific title, these data sets were joined into a single. It
details, a titles release year, its name, who acted in it, the character
they played, its popularity, review score, and so on.

\section{Naming Conventions}\label{naming-conventions}

In examining baby naming trends in the United States, one intriguing
aspect is the persistence of popular names over time. The agency has
tasked me with analyzing the rank correlation, specifically using
Spearman rank correlation, between each year's 25 most popular boys' and
girls' names and those of the subsequent three years. This approach
allows us to assess whether today's favored names continue to dominate
in future years, providing insights into the longevity of naming
trends.As such, this investigation aims to confirm or refute the
agency's suspicion that since the 1990s, popular name trends have shown
reduced persistence compared to earlier decades.

\begin{figure}[H]

{\centering \includegraphics{Question1_files/figure-latex/Figure1-1} 

}

\caption{Spearman Correlations Over Time.\label{Figure1}}\label{fig:Figure1}
\end{figure}

As can be seen from the tables above, there is a clear divergence in
persistence between boys' and girls' naming trends. Boys' names have
exhibited a notable decline in staying power since the 90's when
compared to earlier decades. In other words, boys have a faster turnover
in naming preferences. In contrast, girls' names have shown greater
resilience, consistently maintaining or exceeding their popularity
levels from the 1990s. While girls' names experienced fluctuations with
occasional peaks, they generally reverted to similar levels over time.
When examining both genders collectively, the overall trend reveals a
decline in the persistence of popular names across the board. This
finding underscores shifting societal influences and perhaps evolving
cultural dynamics such as interest in cinematic media.

\section{Cultural Influences}\label{cultural-influences}

We see further in the figures below, showing that the top five most
popular names overall have all past their peaks for both males and
females and that naming conventions, while sticky, do not persist
indefinitely. While their are likely many cultural and sociological
factors that impact on this change, media is likely one of them.

\begin{figure}[H]

{\centering \includegraphics{Question1_files/figure-latex/Figure2-1} 

}

\caption{Top 5 Male Names Over Time.\label{Figure2}}\label{fig:Figure2}
\end{figure}

\begin{figure}[H]

{\centering \includegraphics{Question1_files/figure-latex/Figure3-1} 

}

\caption{Top 5 Male Names Over Time.\label{Figure3}}\label{fig:Figure3}
\end{figure}

We examine this by first joining the baby names data set to the actors
and titles data set according to actor name and the titles release year.
This gives us the count of a particular name in accordance with the name
of the actor in the title for that titles release year. Using this data
set, a dummy variable was created to indicate whether an actor with one
of the top 5 most popular names for males and females was present.
Subsequently, this dummy variable, along with a titles popularity and
review score, was regressed using on the count of a name in the titles
release year using Ordinary Least Squares (POLS) to examine its impact.

 
  \providecommand{\huxb}[2]{\arrayrulecolor[RGB]{#1}\global\arrayrulewidth=#2pt}
  \providecommand{\huxvb}[2]{\color[RGB]{#1}\vrule width #2pt}
  \providecommand{\huxtpad}[1]{\rule{0pt}{#1}}
  \providecommand{\huxbpad}[1]{\rule[-#1]{0pt}{#1}}

\begin{table}[ht]
\begin{centerbox}
\begin{threeparttable}
 \setlength{\tabcolsep}{0pt}
\begin{tabular}{l l l l}


\hhline{>{\huxb{0, 0, 0}{0.8}}->{\huxb{0, 0, 0}{0.8}}->{\huxb{0, 0, 0}{0.8}}->{\huxb{0, 0, 0}{0.8}}-}
\arrayrulecolor{black}

\multicolumn{1}{!{\huxvb{0, 0, 0}{0}}c!{\huxvb{0, 0, 0}{0}}}{\huxtpad{6pt + 1em}\centering \hspace{6pt}  \hspace{6pt}\huxbpad{6pt}} &
\multicolumn{1}{c!{\huxvb{0, 0, 0}{0}}}{\huxtpad{6pt + 1em}\centering \hspace{6pt} POLS \hspace{6pt}\huxbpad{6pt}} &
\multicolumn{1}{c!{\huxvb{0, 0, 0}{0}}}{\huxtpad{6pt + 1em}\centering \hspace{6pt} RE \hspace{6pt}\huxbpad{6pt}} &
\multicolumn{1}{c!{\huxvb{0, 0, 0}{0}}}{\huxtpad{6pt + 1em}\centering \hspace{6pt} FE \hspace{6pt}\huxbpad{6pt}} \tabularnewline[-0.5pt]


\hhline{>{\huxb{255, 255, 255}{0.4}}->{\huxb{0, 0, 0}{0.4}}->{\huxb{0, 0, 0}{0.4}}->{\huxb{0, 0, 0}{0.4}}-}
\arrayrulecolor{black}

\multicolumn{1}{!{\huxvb{0, 0, 0}{0}}l!{\huxvb{0, 0, 0}{0}}}{\huxtpad{6pt + 1em}\raggedright \hspace{6pt} (Intercept) \hspace{6pt}\huxbpad{6pt}} &
\multicolumn{1}{r!{\huxvb{0, 0, 0}{0}}}{\huxtpad{6pt + 1em}\raggedleft \hspace{6pt} -57.228\hphantom{0}\hphantom{0}\hphantom{0}\hphantom{0} \hspace{6pt}\huxbpad{6pt}} &
\multicolumn{1}{r!{\huxvb{0, 0, 0}{0}}}{\huxtpad{6pt + 1em}\raggedleft \hspace{6pt} 871.472\hphantom{0}\hphantom{0}\hphantom{0}\hphantom{0} \hspace{6pt}\huxbpad{6pt}} &
\multicolumn{1}{r!{\huxvb{0, 0, 0}{0}}}{\huxtpad{6pt + 1em}\raggedleft \hspace{6pt} \hphantom{0}\hphantom{0}\hphantom{0}\hphantom{0}\hphantom{0}\hphantom{0}\hphantom{0}\hphantom{0} \hspace{6pt}\huxbpad{6pt}} \tabularnewline[-0.5pt]


\hhline{}
\arrayrulecolor{black}

\multicolumn{1}{!{\huxvb{0, 0, 0}{0}}l!{\huxvb{0, 0, 0}{0}}}{\huxtpad{6pt + 1em}\raggedright \hspace{6pt}  \hspace{6pt}\huxbpad{6pt}} &
\multicolumn{1}{r!{\huxvb{0, 0, 0}{0}}}{\huxtpad{6pt + 1em}\raggedleft \hspace{6pt} (401.237)\hphantom{0}\hphantom{0}\hphantom{0} \hspace{6pt}\huxbpad{6pt}} &
\multicolumn{1}{r!{\huxvb{0, 0, 0}{0}}}{\huxtpad{6pt + 1em}\raggedleft \hspace{6pt} (674.504)\hphantom{0}\hphantom{0}\hphantom{0} \hspace{6pt}\huxbpad{6pt}} &
\multicolumn{1}{r!{\huxvb{0, 0, 0}{0}}}{\huxtpad{6pt + 1em}\raggedleft \hspace{6pt} \hphantom{0}\hphantom{0}\hphantom{0}\hphantom{0}\hphantom{0}\hphantom{0}\hphantom{0}\hphantom{0} \hspace{6pt}\huxbpad{6pt}} \tabularnewline[-0.5pt]


\hhline{}
\arrayrulecolor{black}

\multicolumn{1}{!{\huxvb{0, 0, 0}{0}}l!{\huxvb{0, 0, 0}{0}}}{\huxtpad{6pt + 1em}\raggedright \hspace{6pt} tmdb\_popularity \hspace{6pt}\huxbpad{6pt}} &
\multicolumn{1}{r!{\huxvb{0, 0, 0}{0}}}{\huxtpad{6pt + 1em}\raggedleft \hspace{6pt} -26.904 *** \hspace{6pt}\huxbpad{6pt}} &
\multicolumn{1}{r!{\huxvb{0, 0, 0}{0}}}{\huxtpad{6pt + 1em}\raggedleft \hspace{6pt} -24.013 *** \hspace{6pt}\huxbpad{6pt}} &
\multicolumn{1}{r!{\huxvb{0, 0, 0}{0}}}{\huxtpad{6pt + 1em}\raggedleft \hspace{6pt} \hphantom{0}\hphantom{0}\hphantom{0}\hphantom{0}\hphantom{0}\hphantom{0}\hphantom{0}\hphantom{0} \hspace{6pt}\huxbpad{6pt}} \tabularnewline[-0.5pt]


\hhline{}
\arrayrulecolor{black}

\multicolumn{1}{!{\huxvb{0, 0, 0}{0}}l!{\huxvb{0, 0, 0}{0}}}{\huxtpad{6pt + 1em}\raggedright \hspace{6pt}  \hspace{6pt}\huxbpad{6pt}} &
\multicolumn{1}{r!{\huxvb{0, 0, 0}{0}}}{\huxtpad{6pt + 1em}\raggedleft \hspace{6pt} (1.787)\hphantom{0}\hphantom{0}\hphantom{0} \hspace{6pt}\huxbpad{6pt}} &
\multicolumn{1}{r!{\huxvb{0, 0, 0}{0}}}{\huxtpad{6pt + 1em}\raggedleft \hspace{6pt} (3.238)\hphantom{0}\hphantom{0}\hphantom{0} \hspace{6pt}\huxbpad{6pt}} &
\multicolumn{1}{r!{\huxvb{0, 0, 0}{0}}}{\huxtpad{6pt + 1em}\raggedleft \hspace{6pt} \hphantom{0}\hphantom{0}\hphantom{0}\hphantom{0}\hphantom{0}\hphantom{0}\hphantom{0}\hphantom{0} \hspace{6pt}\huxbpad{6pt}} \tabularnewline[-0.5pt]


\hhline{}
\arrayrulecolor{black}

\multicolumn{1}{!{\huxvb{0, 0, 0}{0}}l!{\huxvb{0, 0, 0}{0}}}{\huxtpad{6pt + 1em}\raggedright \hspace{6pt} imdb\_score \hspace{6pt}\huxbpad{6pt}} &
\multicolumn{1}{r!{\huxvb{0, 0, 0}{0}}}{\huxtpad{6pt + 1em}\raggedleft \hspace{6pt} 730.198 *** \hspace{6pt}\huxbpad{6pt}} &
\multicolumn{1}{r!{\huxvb{0, 0, 0}{0}}}{\huxtpad{6pt + 1em}\raggedleft \hspace{6pt} 555.109 *** \hspace{6pt}\huxbpad{6pt}} &
\multicolumn{1}{r!{\huxvb{0, 0, 0}{0}}}{\huxtpad{6pt + 1em}\raggedleft \hspace{6pt} \hphantom{0}\hphantom{0}\hphantom{0}\hphantom{0}\hphantom{0}\hphantom{0}\hphantom{0}\hphantom{0} \hspace{6pt}\huxbpad{6pt}} \tabularnewline[-0.5pt]


\hhline{}
\arrayrulecolor{black}

\multicolumn{1}{!{\huxvb{0, 0, 0}{0}}l!{\huxvb{0, 0, 0}{0}}}{\huxtpad{6pt + 1em}\raggedright \hspace{6pt}  \hspace{6pt}\huxbpad{6pt}} &
\multicolumn{1}{r!{\huxvb{0, 0, 0}{0}}}{\huxtpad{6pt + 1em}\raggedleft \hspace{6pt} (58.197)\hphantom{0}\hphantom{0}\hphantom{0} \hspace{6pt}\huxbpad{6pt}} &
\multicolumn{1}{r!{\huxvb{0, 0, 0}{0}}}{\huxtpad{6pt + 1em}\raggedleft \hspace{6pt} (98.131)\hphantom{0}\hphantom{0}\hphantom{0} \hspace{6pt}\huxbpad{6pt}} &
\multicolumn{1}{r!{\huxvb{0, 0, 0}{0}}}{\huxtpad{6pt + 1em}\raggedleft \hspace{6pt} \hphantom{0}\hphantom{0}\hphantom{0}\hphantom{0}\hphantom{0}\hphantom{0}\hphantom{0}\hphantom{0} \hspace{6pt}\huxbpad{6pt}} \tabularnewline[-0.5pt]


\hhline{}
\arrayrulecolor{black}

\multicolumn{1}{!{\huxvb{0, 0, 0}{0}}l!{\huxvb{0, 0, 0}{0}}}{\huxtpad{6pt + 1em}\raggedright \hspace{6pt} Top\_5\_actor \hspace{6pt}\huxbpad{6pt}} &
\multicolumn{1}{r!{\huxvb{0, 0, 0}{0}}}{\huxtpad{6pt + 1em}\raggedleft \hspace{6pt} 28911.165 *** \hspace{6pt}\huxbpad{6pt}} &
\multicolumn{1}{r!{\huxvb{0, 0, 0}{0}}}{\huxtpad{6pt + 1em}\raggedleft \hspace{6pt} 28614.871 *** \hspace{6pt}\huxbpad{6pt}} &
\multicolumn{1}{r!{\huxvb{0, 0, 0}{0}}}{\huxtpad{6pt + 1em}\raggedleft \hspace{6pt} 28416.289 *** \hspace{6pt}\huxbpad{6pt}} \tabularnewline[-0.5pt]


\hhline{}
\arrayrulecolor{black}

\multicolumn{1}{!{\huxvb{0, 0, 0}{0}}l!{\huxvb{0, 0, 0}{0}}}{\huxtpad{6pt + 1em}\raggedright \hspace{6pt}  \hspace{6pt}\huxbpad{6pt}} &
\multicolumn{1}{r!{\huxvb{0, 0, 0}{0}}}{\huxtpad{6pt + 1em}\raggedleft \hspace{6pt} (193.715)\hphantom{0}\hphantom{0}\hphantom{0} \hspace{6pt}\huxbpad{6pt}} &
\multicolumn{1}{r!{\huxvb{0, 0, 0}{0}}}{\huxtpad{6pt + 1em}\raggedleft \hspace{6pt} (187.995)\hphantom{0}\hphantom{0}\hphantom{0} \hspace{6pt}\huxbpad{6pt}} &
\multicolumn{1}{r!{\huxvb{0, 0, 0}{0}}}{\huxtpad{6pt + 1em}\raggedleft \hspace{6pt} (191.064)\hphantom{0}\hphantom{0}\hphantom{0} \hspace{6pt}\huxbpad{6pt}} \tabularnewline[-0.5pt]


\hhline{>{\huxb{255, 255, 255}{0.4}}->{\huxb{0, 0, 0}{0.4}}->{\huxb{0, 0, 0}{0.4}}->{\huxb{0, 0, 0}{0.4}}-}
\arrayrulecolor{black}

\multicolumn{1}{!{\huxvb{0, 0, 0}{0}}l!{\huxvb{0, 0, 0}{0}}}{\huxtpad{6pt + 1em}\raggedright \hspace{6pt} N \hspace{6pt}\huxbpad{6pt}} &
\multicolumn{1}{r!{\huxvb{0, 0, 0}{0}}}{\huxtpad{6pt + 1em}\raggedleft \hspace{6pt} 29375\hphantom{0}\hphantom{0}\hphantom{0}\hphantom{0}\hphantom{0}\hphantom{0}\hphantom{0}\hphantom{0} \hspace{6pt}\huxbpad{6pt}} &
\multicolumn{1}{r!{\huxvb{0, 0, 0}{0}}}{\huxtpad{6pt + 1em}\raggedleft \hspace{6pt} 29375\hphantom{0}\hphantom{0}\hphantom{0}\hphantom{0}\hphantom{0}\hphantom{0}\hphantom{0}\hphantom{0} \hspace{6pt}\huxbpad{6pt}} &
\multicolumn{1}{r!{\huxvb{0, 0, 0}{0}}}{\huxtpad{6pt + 1em}\raggedleft \hspace{6pt} 29375\hphantom{0}\hphantom{0}\hphantom{0}\hphantom{0}\hphantom{0}\hphantom{0}\hphantom{0}\hphantom{0} \hspace{6pt}\huxbpad{6pt}} \tabularnewline[-0.5pt]


\hhline{}
\arrayrulecolor{black}

\multicolumn{1}{!{\huxvb{0, 0, 0}{0}}l!{\huxvb{0, 0, 0}{0}}}{\huxtpad{6pt + 1em}\raggedright \hspace{6pt} R2 \hspace{6pt}\huxbpad{6pt}} &
\multicolumn{1}{r!{\huxvb{0, 0, 0}{0}}}{\huxtpad{6pt + 1em}\raggedleft \hspace{6pt} 0.436\hphantom{0}\hphantom{0}\hphantom{0}\hphantom{0} \hspace{6pt}\huxbpad{6pt}} &
\multicolumn{1}{r!{\huxvb{0, 0, 0}{0}}}{\huxtpad{6pt + 1em}\raggedleft \hspace{6pt} 0.443\hphantom{0}\hphantom{0}\hphantom{0}\hphantom{0} \hspace{6pt}\huxbpad{6pt}} &
\multicolumn{1}{r!{\huxvb{0, 0, 0}{0}}}{\huxtpad{6pt + 1em}\raggedleft \hspace{6pt} 0.445\hphantom{0}\hphantom{0}\hphantom{0}\hphantom{0} \hspace{6pt}\huxbpad{6pt}} \tabularnewline[-0.5pt]


\hhline{>{\huxb{0, 0, 0}{0.8}}->{\huxb{0, 0, 0}{0.8}}->{\huxb{0, 0, 0}{0.8}}->{\huxb{0, 0, 0}{0.8}}-}
\arrayrulecolor{black}

\multicolumn{4}{!{\huxvb{0, 0, 0}{0}}l!{\huxvb{0, 0, 0}{0}}}{\huxtpad{6pt + 1em}\raggedright \hspace{6pt}  *** p $<$ 0.001;  ** p $<$ 0.01;  * p $<$ 0.05. \hspace{6pt}\huxbpad{6pt}} \tabularnewline[-0.5pt]


\hhline{}
\arrayrulecolor{black}
\end{tabular}
\end{threeparttable}\par\end{centerbox}

\end{table}
 

A Breusch-Pagan test indicates significant heteroscedasticity,
suggesting variability in error terms across observations, reinforcing
the need for random effects or fixed effects models to account for such
variance. Once accounted for, the results for male names highlight
significant coefficients across all models, all of which indicate that
an actor having one of the top 5 names greatly impacts on name counts
for those names. A Hausman test further indicates that the Fixed Effects
model is preferable due to its lower P-value, suggesting that unobserved
heterogeneity is indeed present and should be accounted for.

 
  \providecommand{\huxb}[2]{\arrayrulecolor[RGB]{#1}\global\arrayrulewidth=#2pt}
  \providecommand{\huxvb}[2]{\color[RGB]{#1}\vrule width #2pt}
  \providecommand{\huxtpad}[1]{\rule{0pt}{#1}}
  \providecommand{\huxbpad}[1]{\rule[-#1]{0pt}{#1}}

\begin{table}[ht]
\begin{centerbox}
\begin{threeparttable}
 \setlength{\tabcolsep}{0pt}
\begin{tabular}{l l l l}


\hhline{>{\huxb{0, 0, 0}{0.8}}->{\huxb{0, 0, 0}{0.8}}->{\huxb{0, 0, 0}{0.8}}->{\huxb{0, 0, 0}{0.8}}-}
\arrayrulecolor{black}

\multicolumn{1}{!{\huxvb{0, 0, 0}{0}}c!{\huxvb{0, 0, 0}{0}}}{\huxtpad{6pt + 1em}\centering \hspace{6pt}  \hspace{6pt}\huxbpad{6pt}} &
\multicolumn{1}{c!{\huxvb{0, 0, 0}{0}}}{\huxtpad{6pt + 1em}\centering \hspace{6pt} POLS \hspace{6pt}\huxbpad{6pt}} &
\multicolumn{1}{c!{\huxvb{0, 0, 0}{0}}}{\huxtpad{6pt + 1em}\centering \hspace{6pt} RE \hspace{6pt}\huxbpad{6pt}} &
\multicolumn{1}{c!{\huxvb{0, 0, 0}{0}}}{\huxtpad{6pt + 1em}\centering \hspace{6pt} FE \hspace{6pt}\huxbpad{6pt}} \tabularnewline[-0.5pt]


\hhline{>{\huxb{255, 255, 255}{0.4}}->{\huxb{0, 0, 0}{0.4}}->{\huxb{0, 0, 0}{0.4}}->{\huxb{0, 0, 0}{0.4}}-}
\arrayrulecolor{black}

\multicolumn{1}{!{\huxvb{0, 0, 0}{0}}l!{\huxvb{0, 0, 0}{0}}}{\huxtpad{6pt + 1em}\raggedright \hspace{6pt} (Intercept) \hspace{6pt}\huxbpad{6pt}} &
\multicolumn{1}{r!{\huxvb{0, 0, 0}{0}}}{\huxtpad{6pt + 1em}\raggedleft \hspace{6pt} 639.161 **\hphantom{0} \hspace{6pt}\huxbpad{6pt}} &
\multicolumn{1}{r!{\huxvb{0, 0, 0}{0}}}{\huxtpad{6pt + 1em}\raggedleft \hspace{6pt} 657.393 **\hphantom{0} \hspace{6pt}\huxbpad{6pt}} &
\multicolumn{1}{r!{\huxvb{0, 0, 0}{0}}}{\huxtpad{6pt + 1em}\raggedleft \hspace{6pt} \hphantom{0}\hphantom{0}\hphantom{0}\hphantom{0}\hphantom{0}\hphantom{0}\hphantom{0}\hphantom{0} \hspace{6pt}\huxbpad{6pt}} \tabularnewline[-0.5pt]


\hhline{}
\arrayrulecolor{black}

\multicolumn{1}{!{\huxvb{0, 0, 0}{0}}l!{\huxvb{0, 0, 0}{0}}}{\huxtpad{6pt + 1em}\raggedright \hspace{6pt}  \hspace{6pt}\huxbpad{6pt}} &
\multicolumn{1}{r!{\huxvb{0, 0, 0}{0}}}{\huxtpad{6pt + 1em}\raggedleft \hspace{6pt} (216.010)\hphantom{0}\hphantom{0}\hphantom{0} \hspace{6pt}\huxbpad{6pt}} &
\multicolumn{1}{r!{\huxvb{0, 0, 0}{0}}}{\huxtpad{6pt + 1em}\raggedleft \hspace{6pt} (238.025)\hphantom{0}\hphantom{0}\hphantom{0} \hspace{6pt}\huxbpad{6pt}} &
\multicolumn{1}{r!{\huxvb{0, 0, 0}{0}}}{\huxtpad{6pt + 1em}\raggedleft \hspace{6pt} \hphantom{0}\hphantom{0}\hphantom{0}\hphantom{0}\hphantom{0}\hphantom{0}\hphantom{0}\hphantom{0} \hspace{6pt}\huxbpad{6pt}} \tabularnewline[-0.5pt]


\hhline{}
\arrayrulecolor{black}

\multicolumn{1}{!{\huxvb{0, 0, 0}{0}}l!{\huxvb{0, 0, 0}{0}}}{\huxtpad{6pt + 1em}\raggedright \hspace{6pt} tmdb\_popularity \hspace{6pt}\huxbpad{6pt}} &
\multicolumn{1}{r!{\huxvb{0, 0, 0}{0}}}{\huxtpad{6pt + 1em}\raggedleft \hspace{6pt} -1.819\hphantom{0}\hphantom{0}\hphantom{0}\hphantom{0} \hspace{6pt}\huxbpad{6pt}} &
\multicolumn{1}{r!{\huxvb{0, 0, 0}{0}}}{\huxtpad{6pt + 1em}\raggedleft \hspace{6pt} -1.883\hphantom{0}\hphantom{0}\hphantom{0}\hphantom{0} \hspace{6pt}\huxbpad{6pt}} &
\multicolumn{1}{r!{\huxvb{0, 0, 0}{0}}}{\huxtpad{6pt + 1em}\raggedleft \hspace{6pt} \hphantom{0}\hphantom{0}\hphantom{0}\hphantom{0}\hphantom{0}\hphantom{0}\hphantom{0}\hphantom{0} \hspace{6pt}\huxbpad{6pt}} \tabularnewline[-0.5pt]


\hhline{}
\arrayrulecolor{black}

\multicolumn{1}{!{\huxvb{0, 0, 0}{0}}l!{\huxvb{0, 0, 0}{0}}}{\huxtpad{6pt + 1em}\raggedright \hspace{6pt}  \hspace{6pt}\huxbpad{6pt}} &
\multicolumn{1}{r!{\huxvb{0, 0, 0}{0}}}{\huxtpad{6pt + 1em}\raggedleft \hspace{6pt} (0.984)\hphantom{0}\hphantom{0}\hphantom{0} \hspace{6pt}\huxbpad{6pt}} &
\multicolumn{1}{r!{\huxvb{0, 0, 0}{0}}}{\huxtpad{6pt + 1em}\raggedleft \hspace{6pt} (1.102)\hphantom{0}\hphantom{0}\hphantom{0} \hspace{6pt}\huxbpad{6pt}} &
\multicolumn{1}{r!{\huxvb{0, 0, 0}{0}}}{\huxtpad{6pt + 1em}\raggedleft \hspace{6pt} \hphantom{0}\hphantom{0}\hphantom{0}\hphantom{0}\hphantom{0}\hphantom{0}\hphantom{0}\hphantom{0} \hspace{6pt}\huxbpad{6pt}} \tabularnewline[-0.5pt]


\hhline{}
\arrayrulecolor{black}

\multicolumn{1}{!{\huxvb{0, 0, 0}{0}}l!{\huxvb{0, 0, 0}{0}}}{\huxtpad{6pt + 1em}\raggedright \hspace{6pt} imdb\_score \hspace{6pt}\huxbpad{6pt}} &
\multicolumn{1}{r!{\huxvb{0, 0, 0}{0}}}{\huxtpad{6pt + 1em}\raggedleft \hspace{6pt} 143.022 *** \hspace{6pt}\huxbpad{6pt}} &
\multicolumn{1}{r!{\huxvb{0, 0, 0}{0}}}{\huxtpad{6pt + 1em}\raggedleft \hspace{6pt} 138.471 *** \hspace{6pt}\huxbpad{6pt}} &
\multicolumn{1}{r!{\huxvb{0, 0, 0}{0}}}{\huxtpad{6pt + 1em}\raggedleft \hspace{6pt} \hphantom{0}\hphantom{0}\hphantom{0}\hphantom{0}\hphantom{0}\hphantom{0}\hphantom{0}\hphantom{0} \hspace{6pt}\huxbpad{6pt}} \tabularnewline[-0.5pt]


\hhline{}
\arrayrulecolor{black}

\multicolumn{1}{!{\huxvb{0, 0, 0}{0}}l!{\huxvb{0, 0, 0}{0}}}{\huxtpad{6pt + 1em}\raggedright \hspace{6pt}  \hspace{6pt}\huxbpad{6pt}} &
\multicolumn{1}{r!{\huxvb{0, 0, 0}{0}}}{\huxtpad{6pt + 1em}\raggedleft \hspace{6pt} (31.426)\hphantom{0}\hphantom{0}\hphantom{0} \hspace{6pt}\huxbpad{6pt}} &
\multicolumn{1}{r!{\huxvb{0, 0, 0}{0}}}{\huxtpad{6pt + 1em}\raggedleft \hspace{6pt} (34.684)\hphantom{0}\hphantom{0}\hphantom{0} \hspace{6pt}\huxbpad{6pt}} &
\multicolumn{1}{r!{\huxvb{0, 0, 0}{0}}}{\huxtpad{6pt + 1em}\raggedleft \hspace{6pt} \hphantom{0}\hphantom{0}\hphantom{0}\hphantom{0}\hphantom{0}\hphantom{0}\hphantom{0}\hphantom{0} \hspace{6pt}\huxbpad{6pt}} \tabularnewline[-0.5pt]


\hhline{}
\arrayrulecolor{black}

\multicolumn{1}{!{\huxvb{0, 0, 0}{0}}l!{\huxvb{0, 0, 0}{0}}}{\huxtpad{6pt + 1em}\raggedright \hspace{6pt} Top\_5\_actor \hspace{6pt}\huxbpad{6pt}} &
\multicolumn{1}{r!{\huxvb{0, 0, 0}{0}}}{\huxtpad{6pt + 1em}\raggedleft \hspace{6pt} 11577.976 *** \hspace{6pt}\huxbpad{6pt}} &
\multicolumn{1}{r!{\huxvb{0, 0, 0}{0}}}{\huxtpad{6pt + 1em}\raggedleft \hspace{6pt} 11570.433 *** \hspace{6pt}\huxbpad{6pt}} &
\multicolumn{1}{r!{\huxvb{0, 0, 0}{0}}}{\huxtpad{6pt + 1em}\raggedleft \hspace{6pt} 11641.610 *** \hspace{6pt}\huxbpad{6pt}} \tabularnewline[-0.5pt]


\hhline{}
\arrayrulecolor{black}

\multicolumn{1}{!{\huxvb{0, 0, 0}{0}}l!{\huxvb{0, 0, 0}{0}}}{\huxtpad{6pt + 1em}\raggedright \hspace{6pt}  \hspace{6pt}\huxbpad{6pt}} &
\multicolumn{1}{r!{\huxvb{0, 0, 0}{0}}}{\huxtpad{6pt + 1em}\raggedleft \hspace{6pt} (207.870)\hphantom{0}\hphantom{0}\hphantom{0} \hspace{6pt}\huxbpad{6pt}} &
\multicolumn{1}{r!{\huxvb{0, 0, 0}{0}}}{\huxtpad{6pt + 1em}\raggedleft \hspace{6pt} (207.399)\hphantom{0}\hphantom{0}\hphantom{0} \hspace{6pt}\huxbpad{6pt}} &
\multicolumn{1}{r!{\huxvb{0, 0, 0}{0}}}{\huxtpad{6pt + 1em}\raggedleft \hspace{6pt} (215.951)\hphantom{0}\hphantom{0}\hphantom{0} \hspace{6pt}\huxbpad{6pt}} \tabularnewline[-0.5pt]


\hhline{>{\huxb{255, 255, 255}{0.4}}->{\huxb{0, 0, 0}{0.4}}->{\huxb{0, 0, 0}{0.4}}->{\huxb{0, 0, 0}{0.4}}-}
\arrayrulecolor{black}

\multicolumn{1}{!{\huxvb{0, 0, 0}{0}}l!{\huxvb{0, 0, 0}{0}}}{\huxtpad{6pt + 1em}\raggedright \hspace{6pt} N \hspace{6pt}\huxbpad{6pt}} &
\multicolumn{1}{r!{\huxvb{0, 0, 0}{0}}}{\huxtpad{6pt + 1em}\raggedleft \hspace{6pt} 20653\hphantom{0}\hphantom{0}\hphantom{0}\hphantom{0}\hphantom{0}\hphantom{0}\hphantom{0}\hphantom{0} \hspace{6pt}\huxbpad{6pt}} &
\multicolumn{1}{r!{\huxvb{0, 0, 0}{0}}}{\huxtpad{6pt + 1em}\raggedleft \hspace{6pt} 20653\hphantom{0}\hphantom{0}\hphantom{0}\hphantom{0}\hphantom{0}\hphantom{0}\hphantom{0}\hphantom{0} \hspace{6pt}\huxbpad{6pt}} &
\multicolumn{1}{r!{\huxvb{0, 0, 0}{0}}}{\huxtpad{6pt + 1em}\raggedleft \hspace{6pt} 20653\hphantom{0}\hphantom{0}\hphantom{0}\hphantom{0}\hphantom{0}\hphantom{0}\hphantom{0}\hphantom{0} \hspace{6pt}\huxbpad{6pt}} \tabularnewline[-0.5pt]


\hhline{}
\arrayrulecolor{black}

\multicolumn{1}{!{\huxvb{0, 0, 0}{0}}l!{\huxvb{0, 0, 0}{0}}}{\huxtpad{6pt + 1em}\raggedright \hspace{6pt} R2 \hspace{6pt}\huxbpad{6pt}} &
\multicolumn{1}{r!{\huxvb{0, 0, 0}{0}}}{\huxtpad{6pt + 1em}\raggedleft \hspace{6pt} 0.131\hphantom{0}\hphantom{0}\hphantom{0}\hphantom{0} \hspace{6pt}\huxbpad{6pt}} &
\multicolumn{1}{r!{\huxvb{0, 0, 0}{0}}}{\huxtpad{6pt + 1em}\raggedleft \hspace{6pt} 0.131\hphantom{0}\hphantom{0}\hphantom{0}\hphantom{0} \hspace{6pt}\huxbpad{6pt}} &
\multicolumn{1}{r!{\huxvb{0, 0, 0}{0}}}{\huxtpad{6pt + 1em}\raggedleft \hspace{6pt} 0.133\hphantom{0}\hphantom{0}\hphantom{0}\hphantom{0} \hspace{6pt}\huxbpad{6pt}} \tabularnewline[-0.5pt]


\hhline{>{\huxb{0, 0, 0}{0.8}}->{\huxb{0, 0, 0}{0.8}}->{\huxb{0, 0, 0}{0.8}}->{\huxb{0, 0, 0}{0.8}}-}
\arrayrulecolor{black}

\multicolumn{4}{!{\huxvb{0, 0, 0}{0}}l!{\huxvb{0, 0, 0}{0}}}{\huxtpad{6pt + 1em}\raggedright \hspace{6pt}  *** p $<$ 0.001;  ** p $<$ 0.01;  * p $<$ 0.05. \hspace{6pt}\huxbpad{6pt}} \tabularnewline[-0.5pt]


\hhline{}
\arrayrulecolor{black}
\end{tabular}
\end{threeparttable}\par\end{centerbox}

\end{table}
 

We see similar results for female names, however in this case a Hausman
test provides a larger P-value, indicating that endogeneity may be
present and that the Random Effects (RE) for controlling unobserved
heterogeneity in this analysis. Despite this, we still see a large
positive impact on name counts when actress's have one of the top 5
names. Overall, for both male and female names, these findings
underscore the substantial impact of actor name popularity naming
conventions, providing valuable insights into consumer preferences.

\section{Conclusion}\label{conclusion}

In conclusion, our exploration into baby naming trends in the United
States has unveiled a dynamic landscape influenced by a multitude of
cultural and societal factors. Our analysis has shown that while overall
naming trends appear to be stabilizing, there are distinct
gender-specific patterns: boys' names exhibit a noticeable decline in
persistence since the 1990s, whereas girls' names demonstrate greater
resilience over time. These trends reflect evolving cultural dynamics
and possibly changing preferences influenced by various media and
cultural icons.

Moreover, our investigation into the impact of actor names on baby
naming conventions has revealed compelling insights. By linking baby
names data with actor credits and title information, we observed
significant correlations between the presence of top 5 actor names and
increased name counts in movie titles. The adoption of Fixed Effects
models, supported by the Hausman test, underscores the importance of
accounting for unobserved heterogeneity in such analyses, enhancing the
validity of our findings.

Overall, these findings underscore the intricate interplay between
cultural influences, media representation, and societal preferences in
shaping baby naming trends. They provide valuable insights for
understanding consumer behavior and cultural dynamics in contemporary
society, highlighting the enduring relevance of popular culture in
influencing naming practices.

\bibliography{Tex/ref}





\end{document}
